%=======================================================================
% SVN: $Id: ice_stdout.tex 5 2005-12-12 17:41:05Z mvr $
%=======================================================================
% Intro to Output data section

The ice model produces three types of output data.  A file containing
ASCII text, also known as a log file, is created for each run that
contains information about how the run was set up and how it progressed.
A series of binary restart files necessary to continue the run are created.
A series of netCDF history files containing gridded instantaneous or
time-averaged output are also generated during a run.  These are described below. 

% Standard output files

\subsection{Stdout Output}
\label{stdout}

Diagnostics from the ice model are written to an ASCII file that contains
information from the compilation, a record
of the input parameters, and how hemispherically averaged, maximum and minimum
values are evolving with the integration.  Certain error conditions detected
within the ice setup script or the ice model will also appear in this file.
Upon the completion of the simulation, some timing information will appear
at the bottom of the file.
The file name is of the form {\bf ice.log.\$LID}, where \texttt{\$LID} is
a timestamp for the file ID. It resides in the executable directory.
The frequency of the diagnostics is determined by the namelist parameter
\texttt{diagfreq}. Other diagnostic messages appear in the {\bf ccsm.log.\$LID}
or {\bf cpl.log.\$LID} files in the executable directory. See the CESM scripts
documentation.

