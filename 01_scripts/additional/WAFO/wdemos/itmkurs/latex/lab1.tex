\renewcommand{\FileId}{File: lab1.tex, Last changed: 2005-03-02}

%%%%%%%%%%%%%%%%%%%%%%%%%%%%%%%%%%%%%%%%%%%%%%%
% Analysis of Load Data
%%%%%%%%%%%%%%%%%%%%%%%%%%%%%%%%%%%%%%%%%%%%%%%

%\cleardoublepage
\chapter{Analysis of Load Data}
\label{lab1}

%%%%%%%%%%%%%%%%%%%%%%%%%%%%%%%%%%
% A Stochastic Load Process
%%%%%%%%%%%%%%%%%%%%%%%%%%%%%%%%%%

\section{Measured data}

Here we will consider a measured wave load from deep water.
The load signal $x(t)$ is the level of the sea-surface (measured in meters)
at a fixed point. A measurement of $x(t)$ is saved in
the file \verb|deep.dat|. The first column contains the time and the
second values of the load.
Plot the whole load and zoom in the first 1000 values
\begin{code}
>> load deep.dat
>> x = deep;
>> plot(x(:,1),x(:,2))
>> plot(x(1:1000,1),x(1:1000,2))
\end{code}
The duration of the measurements in seconds, $T$, is computed by
\begin{code}
>> T=x(end,1)-x(1,1);
\end{code}
You can view the variables in the workspace by typing
\begin{code}
>> whos
\end{code}

Estimate the mean and the standard deviation of $X(t)$. (Hint: Use
\verb|mean| and \verb|std|. Start with \verb|help mean|.)
$$m = \longline,~~\sigma = \longline.$$

When analyzing load data only the sequence of turning points
(i.e.~the sequence of local extremes) is of interest and
not the exact path between the local extremes.
Obtain the turning points
for \verb|deep| and compare with the original time signal:
\begin{code}
>> tp = dat2tp(x);
>> plot(x(:,1),x(:,2),tp(:,1),tp(:,2),'.-')
>> axis([0 100 -20 20])
\end{code}
To store the turning points instead of the original time signal is
a good way to compress load data. When analyzing the power
spectrum of the load, one needs the whole time signal, but when
analyzing the level crossings and the rainflow cycles, then the turning
points yield sufficient information.

It is also possible to apply a rainflow filter (also called hysteresis
filter), which removes small oscillations from the signal. All
rainflow cycles with amplitudes below the
threshold \verb+h+ are removed.
\begin{code}
>> tp1 = dat2tp(x,1);
>> plot(x(:,1),x(:,2),tp(:,1),tp(:,2),tp1(:,1),tp1(:,2))
>> axis([0 100 -20 20])
\end{code}

Next we shall compute the mean frequency $f_0$ and the irregularity
factor $\alpha$, see Introduction for definitions.


First we compute a two column matrix with levels
and number of upcrossings of these levels. Then the crossings will be
divided by the time duration $T$ in order to get the intensity of
crossings; ``how many per time unit (second)''.
\begin{code}
>> lc = tp2lc(tp);
>> lc(:,2)=lc(:,2)/T;
>> plot(lc(:,1),lc(:,2))
>> semilogx(lc(:,2),lc(:,1))
\end{code}
In order to obtain the mean frequency $f_0$ we will use the Matlab function
\verb|interp1|. Type \verb|help interp1| to read about the
routine. (You are recommended to use \verb|help| whenever a new
function or routine is introduced.)
\begin{code}
>> m=mean(x(:,2));
>> f0 = interp1(lc(:,1),lc(:,2),m,'linear');
>> f0
\end{code}
Finally we compute the irregularity factor $\alpha$.
The intensity of local maxima
is equal to the number of local extremes in the sequence of turning points
divided by $2T$, so the parameter $\alpha$ can be computed by
\begin{code}
>> extr0=length(tp)/2/T;
>> alfa=f0/extr0
\end{code}

\section{Gaussian process as a model for the deep water data}
Wave data for deep water is often modelled as a Gaussian process,
see Introduction for definitions and simple properties.
The most important notion is the pdf function\footnote{\textbf{p}robability
  \textbf{d}ensity \textbf{f}unction} for the normal distribution with
mean $m$ and standard deviation $\sigma$ computed
in the previous section.

Using normal probability paper, we can check the agreement between data
and the assumed, normal model. %(Use \verb|normplot|)
\begin{code}
>> wnormplot(x(:,2))
\end{code}
Although the pdf function is important in fatigue analysis
it is more important that the crossing intensity derived from the model
is in agreement with the one observed from the signals, see Computer Exercise 3 for
more detailed discussion.

We shall use Rice's formula, given in the Introduction, to compute the
theoretical crossing intensity for Gaussian processes. It contains the
two spectral moments $\lambda_0$ and  $\lambda_2$ and in order to
compute them we need to estimate the spectrum of the load $x$.
Estimate the spectral density of the deep water data
\begin{code}
>> S = dat2spec(deep);
>> wspecplot(S);
\end{code}
The spectral density $S$ is saved as a Matlab structure containing
some additional information; to observe the structure and plot the
estimated density, just execute
\begin{code}
>> S
>> plot(S.w,S.S)
\end{code}
The spectral moments can be computed from the estimated spectral
density by means of  numerical integration
by using \verb+spec2mom+.
\begin{code}
>> lam = spec2mom(S,4); L0=lam(1); L2=lam(2); L4=lam(3);
\end{code}
The variables \verb+L0+, \verb+L2+, and \verb+L4+ contain the spectral
moments $\lambda_0$, $\lambda_2$, and $\lambda_4$, respectively.
Now we can compare the intensity of level crossings from Rice's
formula with the observed number of level crossings. First we compute
the mean frequency $f_0$, then the crossing intensity function $\mu(u)$.
(Note that we assume that the mean of signal $m$ is zero.)
\begin{code}
>> f0=1/(2*pi)*sqrt(L2/L0)
>> ux = -20:0.1:20;
>> ricex = f0*exp(-ux.*ux./(2*L0));
>> plot(lc(:,1),lc(:,2),'-',ux,ricex,'--')
>> semilogx(lc(:,2),lc(:,1),'-',ricex,ux,'--')
\end{code}



%%%%%%%%%%%%%%%%%%%%%%%%%%%%%%%%%%%%%%%%%
% Cycle counts
%%%%%%%%%%%%%%%%%%%%%%%%%%%%%%%%%%%%%%%%%
%---------------------
\section{Rainflow Cycles}
%---------------------
%Consider the process $X(t)$ of Exercise~\ref{ex2.2}. Simulate and plot two
%sample pathes of $X(t)$ for different values of $\zeta$.
%\begin{code}
%>> L0 = simosc(500*2*pi,0.1,0.01);
%>> L1 = simosc(500*2*pi,0.1,0.9);
%>> subplot(211), plot(L0(1:2000,1),L0(1:2000,2))
%>> subplot(212), plot(L1(1:2000,1),L1(1:2000,2))
%\end{code}
%How many upcrossings of the level~0 can we expect in \verb|L0| and
%\verb|L1|? (Hint: use Rice's formula to compute the intensity of zero upcrossings.)
%$$N_T^+(0) = \longline$$
%Compare it with the number of upcrossings in the loads
%\begin{code}
%>> load2ud([0 0 1],L0)
%>> load2ud([0 0 1],L1)
%\end{code}

Recall the  definition of rainflow and min-max cycle counts.
%What is the sequence of turning points?
The demo program \verb|democc| illustrates these definitions. Use it to identify the first few rainflow and
min-max cycles in \verb|x|.
\begin{code}
>> proc = x(1:500,:);
>> democc
\end{code}
Two windows will appear. In Demonstration Window 1, first mark the
turning points by the button TP. Then choose a local maximum (with
the buttons marked $+1,-1,+5,-5$) and find the corresponding cycle
counts (with the buttons RFC,TP). The cycles are visualized in the
other window.

We shall now examine cycle counts in the load \verb|x|.
From the sequence of turning points \verb|tp| we can find the rainflow
and min-max cycles in the data set
\begin{code}
>> RFC = tp2rfc(tp);
>> mM = tp2mm(tp);
\end{code}
Since each cycle is a pair of a local maximum and a local minimum in the
load, a cycle count can be visualized as a set of pairs in the
$\mbox{\bf R}^2$-plane.
Compare the rainflow and min-max counts in the load.
\begin{code}
>> subplot(1,2,1), ccplot(RFC)
>> subplot(1,2,2), ccplot(mM)
\end{code}
Observe that \verb|RFC| contains more cycles with high amplitudes,
compared to \verb|mM|. This becomes more evident in an amplitude histogram.
\begin{code}
>> ampRFC = cc2amp(RFC);
>> ampmM = cc2amp(mM);
>> subplot(1,2,1), hist(ampRFC)
>> subplot(1,2,2), hist(ampmM)
\end{code}

%%%%%%%%%%%%%%%%%%%%%%%%%%%%%%%%%%%%%%%%%
% Turning Points & Rainflow filter
%%%%%%%%%%%%%%%%%%%%%%%%%%%%%%%%%%%%%%%%%
\subsection{Turning Points \& Rainflow Filter}%@@@@@@@@@@@@@@@@

% Vilka tr�skelvidder f�r rainflow-filtret �r l�mpliga f�r din signal?
% En tumregel �r ca 10% av totala vidden.
% Prova n�gra olika tr�sklar, och j�mf�r resultatet.
Which threshold ranges are appropriate for our signal?  A rule of
thumb is about 10\% of the total range.  Try some thresholds, and
compare the results.  How large reduction do we obtain?  How much
damage is kept in the signal?
\begin{code}
>> h1=2; h2=5; h3=10;            % Threshold ranges for the rainflow filter
>> tp_0 = dat2tp(x);               % No rainflow filter
>> tp_1 = dat2tp(x,h1);            % Rainflow filter, h1
>> tp_2 = dat2tp(x,h2);            % Rainflow filter, h2
>> tp_3 = dat2tp(x,h3);            % Rainflow filter, h3

>> whos   % How large reduction in number of cycles?

% How much damage do we loose?
>> beta = 5; % Define a damage exponent
>> dam_0 = cc2dam(tp2rfc(tp_0,'CS'),beta); dam_1 = cc2dam(tp2rfc(tp_1,'CS'),beta);
>> cc2dam(tp2rfc(tp_1,'CS'),beta); dam_2 = cc2dam(tp2rfc(tp_2,'CS'),beta);
>> cc2dam(tp2rfc(tp_2,'CS'),beta); dam_3 = cc2dam(tp2rfc(tp_3,'CS'),beta);

>> [dam_1 dam_2 dam_3]             % Damage
>> [dam_1 dam_2 dam_3]/dam_0       % Relative damage
\end{code}
% Fr�gor:
% - Hur stor reduktion av antalet cykler fick vi?
% - Hur stor del av skadan bibeh�lls?
% - Vilket tr�skelv�rde vill du v�lja?
Questions:
\begin{itemize}
    \item How large reduction in number of cycles did you obtain?
    \item How much of the damage was kept?
    \item Which threshold would you like to chose?
\end{itemize}
% V�lj ett tr�skelv�rde!
Choose a threshold value.
\begin{code}
>> h = ... your choice ...

>> tp = dat2tp(x,h);               % Rainflow filter
>> rfc = tp2rfc(tp,'CS');          % Rainflow cycles
>> dam = cc2dam(rfc,beta);         % Damage

>> dam/dam_0                       % Relative damage
>> length(tp_0)                    % Number of turning points
>> length(tp)                      % Number of TP after rainflow filter
>> length(tp_0)/length(tp)         % Relative length

>> plot(x(:,1),x(:,2),tp(:,1),tp(:,2)) % Compare signals before/after rainflow filter
\end{code}


%%%%%%%%%%%%%%%%%%%%%%%%%%%%%%%%%%%%%%%%%
% Rainflow Matrix
%%%%%%%%%%%%%%%%%%%%%%%%%%%%%%%%%%%%%%%%%
\subsection{Rainflow Matrix}%@@@@@@@@@@@@@@@@

There are different ways of plotting the rainflow matrix.
\begin{code}
>> n = 64;                         % Number of discrete levels
>> [RFM,u,param] = dat2rfm(tp,h,n);% Rainflow matrisx

% Draw the rainflow matrix in Min-Max-format
>> cmatplot(u,u,RFM,3), colorbar

% Draw the rainflow matrix in Mean-Amplitude-format
>> [RFMrm,paramM,paramR,paramA] = cmat2rmcmat(RFM,param);
>> ua=levels(paramA); um=levels(paramM); cmatplot(um,ua,RFMrm',3), colorbar
>> xlabel('Mean'), ylabel('Amplitude')

% It is also possible to define the discretization levels directly
>> param = [-150 150 100]; % Define discretization
>> [RFM,u,param] = dat2rfm(tp,h,param);

% Draw the rainflow matrix in Min-Max-format
>> cmatplot(u,u,RFM,3), colorbar
\end{code}

%--------------------------------------
% Niv�korsningar
% Hur ber�knas niv�korsningarna fr�n rainflowmatrisen?
From the rainflow matrix the level crossings can be obtained.  How
is the level crossings calculated from the rainflow matrix?
\begin{code}
>> lc = cmat2lc(param,RFM);        % Calculate the level crossing spectrum
% Plot the load spectrum in different ways
>> figure(2),
>> plot(lc(:,1),lc(:,2))           % Frequency function
>> semilogy(lc(:,1),lc(:,2))       % Frequency function (log-scale)
>> semilogx(lc(:,2),lc(:,1))       % The fatigue way of plotting
\end{code}

%--------------------------------------
% Lastspektrum (f�r rainflow cykler)

%The rainflow amplitude histogram is obtained by summing over the
%cycle mean values in the rainflow matrix.  How?
The rainflow amplitude histogram can obtained from the rainflow
matrix.  How?
\begin{code}
>> amp = cmat2amp(param,RFM);      % Calculate the amplitude histogram
>> figure(3)
>> plot(amp(:,1),amp(:,2));        % Plot the frequency function
>> semilogy(amp(:,1),amp(:,2),'*');% Frequency function in log-scale
\end{code}
% Med  "lastspektrum" menas ofta det kumulativa antalet cykler
% som funktion av amplituden
The load spectrum is the most common way to present the rainflow
amplitudes, where the cumulative number of cycles above a certain
amplitude is plotted versus the amplitude, i.e. the load spectrum
is the survival function.
\begin{code}
>> lsplot(amp);                    % Cumulative number of cycles
>> lsplot(amp,0,0);                % Histogram of the number of cycles
>> lsplot(amp,0,0,beta);           % Damage histogram
\end{code}


%%%%%%%%%%%%%%%%%%%%%%%%%%%%%%%%%%%%%%%%%
% SN-data
%%%%%%%%%%%%%%%%%%%%%%%%%%%%%%%%%%%%%%%%%
\subsection{Calculation of damage intensity}%@@@@@@@@@@@@@@@@
In the section with optional exercises
one can estimate parameters in the S-N curve.
The estimated parameters are:
$\gamma=5.5\cdot10^{-10}$, $\beta=3.2$, and $\sigma^2_K=0.06$.
These numerical values will be used in the examples below.
For our load $x$ the intensity is estimated as follows
\begin{code}
>> beta=3.2; gam=5.5E-10;
>> d_beta=cc2dam(RFC,beta)/T;
>> time_fail=1/gam/d_beta/3600 %in hours of the specific storm
\end{code}


\section{Additional exercises, Optional}

In the following exercises we shall use a slightly different parameterization
of the S-N curve than given in Introduction, viz.
\begin{equation}\label{SNmodel}
        N(s)=\left\{ \begin{array}{c@{\quad}l}
        K^{-1} \epsilon^{-1}s^{-\beta} & s> s_{\infty},\\
        \infty & s\le s_{\infty},\end{array}\right.
\end{equation}
where $K$ is a  lognormally distributed random factor,
i.e. $\mbox{ln}(K)\in\mbox{N}(0,\sigma^2_K)$,
and $\epsilon$, $\beta$ are fixed material dependent constants.

\subsection{Estimation of S-N curve}
Taking the logarithm of Eq.~(\ref{SNmodel}) and assuming that
$\mbox{ln}(K)\in\mbox{N}(0,\sigma^2_K)$ we obtain
        \begin{equation}\label{logN}
        \mbox{ln}(N(s))=-\mbox{ln}(K)-\mbox{ln}(\epsilon)-\beta\,\mbox{ln}(s)
        \in\mbox{N}(-\mbox{ln}(\epsilon)-\beta\,\mbox{ln}(s),\sigma^2_K),
        \end{equation}
for every fixed $s>s_{\infty}$, $\epsilon$ and $\beta$.

Let $T$ be the fatigue life time. Since the frequency of the load
oscillation $\omega$ is constant we have
        $$\Pr[\,T\le t\,]=\Pr[\,N(s)\le {\textstyle\frac{\omega}{2\pi}}t\,]=
                \Pr[\,K\le \epsilon s^\beta{\textstyle\frac{\omega}{2\pi}}t\,],$$
where $\frac{\omega}{2\pi}t$ is the number of cycles in the interval $[0,t]$.

In the following exercises we shall estimate the parameters in the model
(\ref{SNmodel}).

Load the SN-data by typing
\begin{code}
>> load SN
\end{code}
There are two variables \verb|s| and \verb|N| representing
$s$ and $N(s)$. Plot $N(s)$ against $s$ by
\begin{code}
>> plot(N,s,'o')
>> axis([0 14e5 5 35 ])
>> loglog(N,s,'o')
\end{code}

In the following we assume that $s_{\infty}=0$.

The plotted data consist of 5
groups at $s$~=~10, 15, 20, 25 and 30 MPa. Each group has 8 observations of
$N(s)$ making a total of 40 observations.
Assume that the observations are independent, that the model (\ref{SNmodel})
holds and $K$ is a lognormal variable.


\begin{enumerate}
\item Propose an estimation procedure for $\epsilon$, $\beta$ and $\sigma_K^2$.
\item Check the applicability of (\ref{SNmodel}) by using normal
probability paper,
\begin{code}
>> wnormplot(reshape(log(N),8,5))
\end{code}
\item Use \verb|snplot| to get estimates of
$\epsilon$, $\beta$ and $\sigma^2_K$. (Try \verb+help snplot+.)
        $$\begin{tabular}{lll}
        $\epsilon\approx$ \shortline, &
        $\beta\approx$    \shortline, &
        $\sigma^2_K\approx$ \shortline.
        \end{tabular}$$

{\bf Solution:}
\begin{code}
>> [e0,beta0,s20] = snplot(s,N,12)
>> [e0,beta0,s20] = snplot(s,N,14)
e0    = 5.5361e-10
beta0 = 3.2286
s20   = 0.0604
\end{code}

\subsection{Calculation of the 95\% quantile for the fatigue life time}
  Estimate $t_{0.95}$ defined by
        $$\Pr[\,T>t_{0.95}\,]=0.95,$$
        for $s=22$ MPa and $\omega=10\cdot2\pi$.
\tv\par{\bf Solution:}
We want to solve the equation
        $$\alpha=\Pr[\,T>t_\alpha\,]=1-\Pr[\,T\le t_\alpha\,].$$
Since
        $$\{T\le t_\alpha\}\quad\Leftrightarrow\quad
                \{N(s)\le {\textstyle\frac{\omega}{2\pi}}t_\alpha\}$$
we have
        \begin{eqnarray*}
        \alpha&=&1-\Pr[\,T\le t_\alpha\,]=1-\Pr[\,N(s)\le
                  {\textstyle\frac{\omega}{2\pi}}t_\alpha\,]
                =1-\Pr[\,{\textstyle\frac{K}{\epsilon s^\beta}}\le
                        {\textstyle\frac{\omega}{2\pi}}t_\alpha\,]\\
        &=& 1 - \Pr[\,K\le {\textstyle \epsilon s^\beta  \frac{\omega
                        }{2\pi}t_\alpha}\,]
                =1-\Phi\biggl(
                \frac{\mbox{ln}(\epsilon s^\beta\omega t_\alpha/(2\pi))}
                        {\sigma_K}\biggr)
        \end{eqnarray*}
which gives
        $$\frac{\mbox{ln}(\frac{\epsilon s^\beta\omega t_\alpha}{2\pi})}
                        {\sigma_{\!K}}= \lambda_\alpha\quad
        \Leftrightarrow\quad
        t_\alpha=\frac{2\pi \, \mbox{exp}(\lambda_\alpha\sigma_{\!K})}
                {\epsilon s^\beta\omega}$$
where $\lambda_{\alpha}$ is the $\alpha$-quantile of N(0,1),
i.e.\ $\Pr(X>\lambda_{\alpha})=\alpha$ where $X\in$ N(0,1).
\fi
\end{enumerate}




\subsection{Fatigue life distribution under variable random load}

Compare the total damage caused by rainflow cycles for loads \verb|L1|
and \verb|L2|.
\begin{code}
>> D0 = e0*cumsum((RFC(:,2)-RFC(:,1)).^beta0);
>> plot(D0)
\end{code}

Let $T^f$ be the fatigue failure time.
The failure time distribution is computed as follows
        $$\Pr[\,T^f\le t\,]=\Pr[\,D(t)\ge1\,]=\Pr[\,K\le\epsilon
        D_\beta(t)\,],$$
where $D_{\beta}(t)$ is defined by (\ref{Damage}).
For loads with short memory the damage $D_\beta(t)$ is
asymptotically Gaussian, i.e.
        $$D_\beta(t) \approx \mbox{N}(d_\beta t,\sigma_\beta^2
        t),\qquad
        \mbox{for large values of $t$.}
        $$
where $d_\beta$ is called {\em the damage intensity}
        $$d_\beta=\lim_{t\to\infty}\frac{D_\beta(t)}{t}\qquad\mbox{and}\qquad
        \sigma_\beta^2=\lim_{t\to\infty}\frac{\V[\,D_\beta(t)\,]}{t}.$$
Since $\epsilon$ is small and $\mbox{ln} K\in\mbox{N}(0,\sigma^2)$
        $$F_{T^f}(t)=\Pr[\,T^f\le t\,]=\Pr[\,D(t)\ge1\,]
          = \Pr[\,K\le\epsilon D_{\beta}(t)\,] \approx$$
        $$  \approx \Pr\left[\,K\le\epsilon\left(d_{\beta} t+\sigma_{\beta}\sqrt{t}Z\right)\,\right]
        =\int_{-\infty}^{\infty}\!
            \Pr[\,K\le\epsilon(d_{\beta} t+\sigma_{\beta}\sqrt{t}z)\,]
            \,\phi(z)\,\mbox{d}z
          =$$
       \begin{equation} \label{FTF}
         = \int_{-\infty}^{\infty}\!\Phi\left(
                        \frac{\mbox{ln}\, \epsilon+\mbox{ln} \,
                          d_\beta t+ \mbox{ln}(1+
                        \frac{\sigma_{\!\beta}}{d_\beta}\frac{1}{\sqrt{t}}z)}%
                        {\sigma}\right)\,\phi(z)\,\mbox{d}z,
       \end{equation}
where we used that $D_{\beta}(t)\approx d_{\beta} t+\sigma_{\beta}\sqrt{t}Z$, $Z\in \mbox{N}(0,1)$.
If the cycle count $\{(x,y)_{t_i}\}$ is given, then the damage
intensity $d_\beta$ is estimated by using the function \verb|cc2dam|.

Estimate the damage intensity, $d_{\beta}$, (as a function of parameter
$\beta$) due to the rainflow count in load \verb|x|.
\begin{code}
>> beta = 3:0.1:8;
>> DRFC = cc2dam(RFC,beta);
>> dRFC = DRFC/T
>> plot(beta,dRFC)
\end{code}

Recall that for the S-N data
%of Exercise~\ref{ex2}
we have
$\epsilon=5.5\cdot10^{-10}$, $\beta=3.2$ and $\sigma^2=0.06$. Further
we have estimated $\sigma^2_{\!\beta}=0.5$. Estimate the failure
distribution, using formula (\ref{FTF}) implemented in the function \verb|ftf|
\begin{code}
>> help ftf
>> [t0,F0] = ftf(e0,cc2dam(RFC,beta0)/T,s20,0.5,1);
\end{code}
Check the influence of the parameter $\sigma^2_{\!\beta}$ on the
$T^f$-distribution by putting $\sigma^2_{\!\beta}=0$ and
$\sigma^2_{\!\beta}=5$, respectively.
\begin{code}
>> [t1,F1] = ftf(e0,cc2dam(RFC,beta0)/T,s20,0,1);
>> [t2,F2] = ftf(e0,cc2dam(RFC,beta0)/T,s20,5,1);
>> plot(t0,F0,t1,F1,t2,F2)
\end{code}
Since $\epsilon$ is small, $\sigma^2_{\!\beta}$ has little influence on the
$T^f$-distribution and can be omitted, i.e. $\sigma^2_{\!\beta}=0$.

Under the assumption that $\sigma^2_{\!\beta}=0$ compute the
$t_\alpha$ quantile, i.e.
        $$\Pr[\,T^f > t_\alpha\,]=\alpha.$$

{\bf Solution:}
        $t_\alpha=\epsilon^{-1}d_\beta^{-1}\mbox{e}^{\lambda_\alpha\sigma}$
where $\lambda_\alpha$ is the $\alpha$-quantile of $\mbox{N}(0,1)$-distribution.


Plot $t_{0.99}$ for $3\le\beta\le8$ and rainflow count \verb|RFC|,
 and min-max count \verb|mM|.
\begin{code}
>> taRFC = exp(-1.96*sqrt(0.06))/e0./dRFC;
>> DmM = cc2dam(mM,beta);
>> dmM = DmM/T
>> tamM = exp(-1.96*sqrt(0.06))/e0./dmM;
>> plot(beta,taRFC,beta,tamM,'r')
\end{code}





\subsection{Crack growth data}  %%%%%%%%%%%%%




In some applications the degradation of material is defined as the length
of a crack. The strength of a material is assumed to be zero when the
length of the crack reaches a critical level $a_{\mbox{\footnotesize crt}}$.
In laboratory experiments one is subjecting a specimen to a constant
amplitude load. %, see Section \ref{constantampload} on SN-data.
The length of a crack as a function of the number of periods is recorded.
A set of crack length data with very high accuracy of measurement was
presented by Virkler et al.~\cite{Virkler79.A1} in 1979. We shall briefly analyse this
data set.

Load Virkler data by
\begin{code}
>> clear
>> load virkler
\end{code}
The material is saved as a 164 $\times$ 69 matrix with the crack length in the
first column and the number of the cycles for the 68 specimens in the
following columns.
Plot the first column against the second by
\begin{code}
>> plot(v(:,2),v(:,1))
\end{code}
The figure shows typical non-linear character of crack growth phenomena.
Plot all 68 data series on one plot by
\begin{code}
>> plot(v(:,2:69),v(:,1),'b-')
\end{code}

Define the life time $T_a$ as the number of cycles needed to get a crack
with length $a$.
For each specimen one obtains an independent observation of $T_a$
defined as the number of cycles when the crack growth curve crosses
the level $a$.

Use the function \verb|alevel| to get the life time $T_a$ for
$a=15$ by
\begin{code}
>> N = alevel(v,15);
\end{code}
and view the material graphically by
\begin{code}
>> plot(N,ones(1,length(N)),'o')
\end{code}
\begin{enumerate}
\item For a fixed level $a$ = 20, choose an appropriate model for the life time
distribution $T_a$. Check extreme value, lognormal,  etc. Use the
commands \verb|wnormplot|, \verb|gumbelplot|, and \verb|weibplot|. Which distribution
gives a good fit?
%Does the strength (quality) of the material follow
%Example 2 of Chapter 2 in the textbook, the weakest- or
%strongest-link-principle?

\item Let $a_N$ denote the crack length after $N$ cycles. From each
  specimen we can get an observation of $a_N$. The function
  \verb|nlevel| returns the crack length of the
  specimens after a
  specified number of cycles, here with $N=2 \cdot 10^5$.
\begin{code}
>> a = nlevel(v,2e5);
>> plot(a,ones(1,length(a)),'o')
\end{code}
  Find a model for the distribution of the crack length---check the proposed
model in Example 2, Chapter 2 (lognormal).
% EXTRA: Testa olika N-v�rden f�r att f� tag i A, mu, sigma.
  \end{enumerate}
